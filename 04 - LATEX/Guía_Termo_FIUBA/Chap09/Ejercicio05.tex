\section{Ejercicio}\label{ej:Chap09Ejercicio05}
Utilizando los datos dados por la tabla de vapor sobrecalentado, para el estado de presión de $p_1=\SI{5}{kPa(a)}$ y energía interna $u_1=\SI{2588.23}{\frac{kJ}{kg}}$,
\begin{enumerate}
    \item Aproximando el vapor sobrecalentado a un gas ideal, determinar el valor de su constante $R_p=\frac{p_1\,v_1}{T_1}$, válida como constante particular de gas ideal para el vapor de agua alrededor de ese estado.
    \item Calcular el volumen específico de un estado de vapor sobrecalentado a $\SI{20}{kPa(a)}$ y $\SI{200}{\celsius}$ como gas ideal con el $R_p$ calculado en el punto anterior, y compararlo con el valor dado por la tabla de vapor. Determinar el error relativo de la aproximación.
    \item Ídem para un estado a $p=\SI{10}{MPa(a)}$ y $T=\SI{350}{\celsius}$. Determinar el error relativo de la aproximación.
\end{enumerate}