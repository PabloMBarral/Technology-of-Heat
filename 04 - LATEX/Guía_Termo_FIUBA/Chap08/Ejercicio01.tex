\section{Ejercicio}\label{ej:Chap08Ejercicio01}
Estando el estado muerto (o de referencia) a la temperatura $T_0$ y a la presión $p_0$, escribir las expresiones matemáticas utilizadas para calcular los siguiente casos (se suponen conocidas las funciones $u$, $h$, $s$, $v$, $\omega$ y $z$ de todos los estados considerados:
\begin{enumerate}
    \item La exergía $E_{x_c\,1}$ de un sistema cerrado en el estado $1$, con velocidad $\omega_1$ y altura $z_1$.
    \item La variación de exergía $\Delta E_{x_c}$ de un sistema cerrado que evoluciona reversiblemente desde el estado $1$ al $2_s$. Despreciar energías cinéticas y potenciales gravitatorias del sistema.
    \item Ídem punto anterior, pero evolucionando de manera irreversible desde $1$ a $2$.
    \item La exergía $E_{x_a\,3}$ de una corriente que circula por una cañería de un estado $3$ de régimen estacionario, con $\omega_3\neq0$ y $z_3=0$.
    \item  La variación de exergía $\Delta E_{x_a}$ de un sistema abierto en régimen estacionario que evoluciona desde una entrada $3$ a una salida $4$. Despreciar energías cinéticas y potenciales gravitatorias.
    \item La exergía $E_{x_a\,5}$ de una corriente que circula por una cañería en un estado $5$ en régimen transitorio.
    \item La variación de exergía $\Delta E_x$ de un sistema abierto en régimen transitorio, de estado inicial $i$, estado final $f$, de un estado de masa de entrada $e$ y de salida $s$.
    \item El rendimiento exergético de un proceso en general.
\end{enumerate}