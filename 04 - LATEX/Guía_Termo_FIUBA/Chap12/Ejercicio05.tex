\section{Ejercicio}\label{ej:Chap12Ejercicio05}
Durante la noche, sobre una superficie opaca y oscura, a cielo abierto se forman gotas de agua. Esta superficie se enfría, entregando calor directamente a la bóveda celeste por radiación. La temperatura ambiente es de $\SI{18}{\celsius}$ y la humedad relativa es del $90\%$.

¿Cuál es la temperatura máxima a la que puede encontrarse esta superficie en estas condiciones de humedad sobre ella?