\section{Ejercicio}\label{ej:Chap07Ejercicio12}
Una corriente de aire de $\SI{1000}{\frac{kg}{h}}$ en régimen estacionario ingresa a un equipo en el estado $t_1=\SI{100}{\celsius}$ y $p_1=\SI{100}{kPa(a)}$. En este se calienta a presión constante hasta $\SI{300}{\celsius}$ y luego se expande adiabáticamente hasta $\SI{10}{bar(a)}$. Finalmente, se expande según una politrópica que lo deja a $t_4=\SI{100}{\celsius}$ y $p_4=\SI{100}{kPa(a)}$.

Determinar la variación de entropía del sistema (o volumen de control).