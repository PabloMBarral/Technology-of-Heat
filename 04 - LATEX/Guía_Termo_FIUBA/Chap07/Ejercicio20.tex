\section{Ejercicio}\label{ej:Chap07Ejercicio20}
Se quiere mantener refrigerada una cámara a una temperatura constante $t_2=\SI{20}{\celsius}$, para lo cual se extrae una cantidad de calor $Q=\SI{400}{\frac{kJ}{min}}$. Si la atmósfera está a $t_0=\SI{40}{\celsius}$, determinar
\begin{enumerate}
    \item La potencia mínima que consumirá el equipo frigorífico.
    \item El coeficiente de efecto frigorífico $\varepsilon_f$ o $COP_f$.
    \item La variación de entropía de cada fuente.
    \item La variación de entropía del sistema, del medio y del universo.
\end{enumerate}