\section{Ejercicio}\label{ej:Chap04Ejercicio11}

Se desea instalar en una plaza una fuente de agua compuesta por un chorro ascendente. La altura del agua debe ser de $\SI{10}{m}$, contando desde el pico de salida vertical, el cual está al mismo nivel que el espejo de agua. El diámetro del pico es de $d=\SI{30}{mm}$. En todo momento debe considerarse al flujo como incompresible, isotérmico y adiabático. Por otra parte, debe tenerse en cuenta las energías cinéticas, potenciales (gravitatoria), y de flujo (presión).

Determinar
\begin{enumerate}
    \item La velocidad necesaria de salida del agua (despreciando rozamientos).
    \item El caudal másico de agua.
    \item La potencia mínima necesaria de la bomba que toma agua a velocidad despreciable, al nivel del espejo de agua y a la presión atmosférica de $\SI{1}{bar(a)}$, y que la descarga a la altura y presión solicitadas.
\end{enumerate}