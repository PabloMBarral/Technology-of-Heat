\section{Ejercicio}\label{ej:Chap04Ejercicio01}

Escribir las expresiones generales del primer principio apicado a
\begin{enumerate}
    \item Un sistema cerrado que cambia su temperatura, su velocidad y su altura.
    \item Un sistema abierto en régimen estacionario con una corriente de entrada y otra de salida, cuyas temperaturas, presiones, velocidades y alturas son distintas.
    \item Un sistema abierto en régimen transitorio, con dos masas de entrada en los estados $1$ y $2$, y una masa de salida en el estado $3$. Recibe el calor $Q_1$, entrega en calor $Q_2$ y recibe los trabajos $W_1$ y $W_2$.
    \item Un sistema abierto en régimen estacionario con una entrada y una salida, cuyo flujo es isotérmico, adiabático, sin trabajo de circulación (trabajo de eje o trabajo de flecha), incompresible, y sin fricción (cuasiestático). Tener en cuenta velocidades, alturas y presiones. Comenzar el desarrollo partiendo de la expresión obtenida en $2$.
\end{enumerate}