\documentclass[10pt,a4paper]{article}

\usepackage[utf8]{inputenc}
\usepackage[spanish]{babel}
\usepackage[T1]{fontenc}
\usepackage{lmodern}
\usepackage{microtype}

\usepackage{authblk}

\usepackage[cache=false]{minted}
\usemintedstyle{friendly}
\renewcommand\listingscaption{Código}

\usepackage{graphicx}

\usepackage[backref=page, colorlinks=true, citecolor=cyan, linkcolor=blue, urlcolor=blue]{hyperref}
\usepackage[numbered]{bookmark}

%%% No tocar, generado automáticamente %%%
\makeatletter
\def\@fnsymbol#1{\ensuremath{\ifcase#1\or \dagger\or \ddagger\or
   \mathsection\or \mathparagraph\or \|\or **\or \dagger\dagger
   \or \ddagger\ddagger \else\@ctrerr\fi}}
\makeatother
%%%

     \title{Título}
     \author{Ing. Pablo Barral\thanks{Departamento de Ing. Mecánica, Universidad de Buenos Aires; 
     \href{mailto:pbarral@fi.uba.ar}{\texttt{pbarral@fi.uba.ar}};
     \raisebox{0.25ex}{\href{https://www.linkedin.com/in/pablo-barral}{\includegraphics[height=0.75em]{linkedin_logo.png}}}}
     }
     \date{\today}

     
     \hypersetup{
          pdfpagemode=UseOutlines,
          pdftitle={Título del Documento},
          pdfauthor={Ing. Pablo Barral},
          pdfsubject={Asunto del Documento},
          pdfkeywords={keyword1, keyword2, keyword3},
          pdfcreator={LaTeX with hyperref},
          pdfproducer={pdfLaTeX}
          pdfstartview=Fit,
          bookmarksnumbered=true,
          bookmarksopen=true,
          bookmarksopenlevel=1,
     }


\begin{document}
     
     \maketitle
     
\begin{abstract}
     Esto es el resumen del texto. %Poner la version que es. Poner para qué materia.\footnote{Poner acá los nombres.}
\end{abstract}

\vspace{1em}
\noindent{\small\textbf{Palabras clave:} ASME, BPVC, calderas.}

\begin{quote}
     \fbox{%
     \parbox{\dimexpr\linewidth-2\fboxsep}{%
         \textbf{Disclaimer:} El contenido de este apunte tiene como única finalidad ser una introducción somera al código en el marco de una asignatura de grado, apoyándose en una exposición oral. Este apunte no constituye, bajo ningún tipo de concepto, un reemplazo del código, ni en su versión 2023 ni en la última vigente. Este apunte no sustituye de ninguna manera al juicio criterioso en el diseño ni a las reglas del buen arte y la experiencia en el diseño, la fabricación, el ensayo, la inspección y la operación. El autor no asume responsabilidad por ninguna acción tomada basada en la información proporcionada en este apunte, y prohíbe enfáticamente el uso de este apunte para el diseño. Finalmente, se recuerda que el diseño de recipientes a presión sin una validación empírica o siguiendo los lineamientos de un ente reconocido constituye un serio peligro para la seguridad, la integridad de los equipos y las instalaciones y la vida de las personas, con consecuencias posiblemente fatales o incapacitantes de modo permanente.
     }
     }
\end{quote}


%\begin{quote}
%     \fbox{%
%     \parbox{\dimexpr\linewidth-2\fboxsep}{%
%         \textbf{Disclaimer:} The content of this article is for informational purposes only and does not constitute professional advice. The author assumes no responsibility for any actions taken based on the information provided in this article.
%     }
%     }
%\end{quote}
     \section{Introducción}

La transferencia de calor entre la serpentina y la corriente de aire puede calcularse como

     \begin{equation}
          \dot{Q} = \frac{h_c \cdot A}{c_{p,m}}\cdot\left(h_{sat} - h_a \right)
     \end{equation}

Aquí, $h_{sat}$ es la entalpía del aire húmedo a la temperatura de la serpentina. Esta entalpía es saturada, porque el aire húmedo condensa al tocarla, se genera una película de condensado, por lo que el aire húmedo que está en contacto con esa película se satura. La situación es similar a una torre de enfriamiento.

En este caso, la temperatura del metal es menor que la del punto de rocío del aire. Si ese no fuera el caso, entonces toda la serpentina sería de transferencia de calor sensible, pues no habría forma de que condense. Esta serpentina es la común.

La condensación ocurre antes de que la temperatura promedio del aire húmedo llegue a la temperatura de rocío. Esto es porque el aire húmedo que toma contacto con el metal empieza a condensar mucho antes de que el aire que pasa lejos se enfríe por debajo de su punto de rocío. Hay, en este sentido, un gradiente perpendicular al flujo: gradiente de temperatura y de humedad absoulta.

$h_a$ es la humedad del aire lejos de la serpentina.

Estamos siguiendo las secciones 5.7 y 13.1 del libro de Mitchell. También, unas secciones (SM) aparte.

% Mencionar el ADP y el bypass factor.
% Mencionar la cuenta de una AHU, y hacer un análisis de la energía que se lleva el condensado.


\begin{equation}
     m^{\star}=\frac{\dot{m}_a \cdot c_s}{\dot{m}_w \cdot c_{p,w}}
\end{equation}

cs  es un calor específico efectivo.

Es el cambio de la entalpía con respecto a la temperatura a lo largo de la línea de saturación.

Se evalúa con las temperaturas de entrada y salida.

\begin{equation}
     c_s = \left(\frac{h_{w,sat,in}-h_{w,sat,out}}{T_{w,in}-T_{w,out}}\right)
\end{equation}

Tiene unidades de kJ por C y kg (de aire seco). Es la entalpía del aire húmedo, pero se expresa por unidad de masa de aire seco.

Nos interesa ver la entalpía del aire saturado.

m estrella es como un ratio de calores específicos.

Aa es el área que está expuesta al aire.

eta estrella cero es una eficiencia general para la transferencia de masa, y es un valor cercano a la eficiencia en la transferencia de calor.

hc es el coeficiente convectivo

cpm es el calor específico de la mezcla.

\begin{equation}
     U_0^{\ast}\cdot A_a = \frac{\frac{\eta_0^{\ast}\cdot h_c \cdot A_a}{c_{p,m}}}{1+\frac{c_s\cdot \eta^{\ast}_0 \cdot h_c \cdot A_a}{c_{p,m}\cdot U_w \cdot A_w}}
\end{equation}


\begin{equation}
     {Ntu}^{\ast}=\frac{U_0^{\ast}\cdot A_a}{\dot{m}_a}
\end{equation}

Estaría bueno ver las unidades de lo que estoy escribiendo. Especialmente lo de U asterisco.

\begin{equation}
     \dot{Q}=\dot{\varepsilon}\cdot\dot{m}_a\left(h_{a,in}-h_{w,sat,in}\right)
\end{equation}

El calor, en lugar de hacerlo en función de la temperatura, lo hacemos en función a la entalpía.

Estamos asumiendo que vamos a enfriar, porque hay condensación. Por lo que la entalpía del aire de entrada es mayor a la del agua de enfriamiento.

Si no fuera agua, si fuera refrigerante, el análisis sería similar.

El máximo calor sería cuando el aire esté a la temperatura de entrada del agua.

\begin{equation}
     \varepsilon^{\ast}=\frac{\left(h_{a,in}-h_{a,out}\right)}{\left( h_{a,in}-h_{w,sat,in}\right)}
\end{equation}

\begin{equation}
     \dot{Q}=\dot{m}_w \cdot c_{p,w} \cdot \left(T_{w,out}-T_{w,in}\right)
\end{equation}

  
     \newpage
\section{Anexo}
%\inputminted[frame=lines, framesep=2mm,fontsize=\footnotesize,linenos]{python}{code_02.py}    

\subsection{Incluir Archivo Binario}

Puedes descargar el archivo binario haciendo clic en el enlace a continuación:

%\attachfile{graph.EES}
\attachfile[description={ejemplo.EES}]{ejemplo.EES}

\attachfile[description={graph.EES}, icon=Paperclip]{graph.EES}
graph.EES
%% ver si vale la pena incluir los archivos o no

     \begin{thebibliography}{9}
          \bibitem{texbook}
          Donald E. Knuth (1986) \emph{The \TeX{} Book}, Addison-Wesley Professional.
          
          \bibitem{lamport94}
          Leslie Lamport (1994) \emph{\LaTeX: a document preparation system}, Addison
          Wesley, Massachusetts, 2nd ed.
      \end{thebibliography}

\end{document}





1. Uso de tespy, en baja presion y en alta presion, tambien en EES
2. ESCOA
3. Grimision y zukauskas
4. Balance HRSG, con ganapathy
5. Instalaciones de cogeneracion
6. estequiometria
7. calculo de un sobrecalentador, taler
8. calcular un economizador
9. calculo de un desaireador
10. calculo de una perdida de carga
11. metodo de spencer cotton y cannon
12. balance exergetico en una combustion
13. balance exergetico en un ciclo de baja potencia.
14. lo de las fuentes y el aprovechamiento de la exergia

todo lo que haga en EES; hacerlo como procedimiento

hacer chunks de codigo
ver estteem, libro en aleman, ganapathy
el loco de linkedin
cao
el libro de heimo walter
caldera paquete, poner como es el grafico

%%%%%%%%%%%%%%%%%%%%%%%%%%%%%%%%%%

https://hal.science/hal-00699058/document

https://orcid.org/0000-0003-1125-4199

\orcidA{} is used to insert the ORCID icon. Make sure you have \usepackage{academicons} in your preamble to use \orcidA{}.

%%%%%%%%%%%%%%%%%%%%%%%%%%%%%%%%%%



\begin{thebibliography}{999}
     % Aquí va tu bibliografía
     \bibitem[Author1(year)]{ref-journal}
     Author~1, T. The title of the cited article. {\em Journal Abbreviation} {\bf 2008}, {\em 10}, 142--149.
     % Reference 2
     \bibitem[Author2(year)]{ref-book1}
     Author~2, L. The title of the cited contribution. In {\em The Book Title}; Editor1, F., Editor2, A., Eds.; Publishing House: City, Country, 2007; pp. 32--58.
     % Reference 3
     \bibitem[Author3(year)]{ref-book2}
     Author 1, A.; Author 2, B. \textit{Book Title}, 3rd ed.; Publisher: Publisher Location, Country, 2008; pp. 154--196.
     % Reference 4
     \bibitem[Author4(year)]{ref-unpublish}
     Author 1, A.B.; Author 2, C. Title of Unpublished Work. \textit{Abbreviated Journal Name} stage of publication (under review; accepted; in~press).
     % Reference 5
     \bibitem[Author5(year)]{ref-communication}
     Author 1, A.B. (University, City, State, Country); Author 2, C. (Institute, City, State, Country). Personal communication, 2012.
     % Reference 6
     \bibitem[Author6(year)]{ref-proceeding}
     Author 1, A.B.; Author 2, C.D.; Author 3, E.F. Title of Presentation. In Title of the Collected Work (if available), Proceedings of the Name of the Conference, Location of Conference, Country, Date of Conference; Editor 1, Editor 2, Eds. (if available); Publisher: City, Country, Year (if available); Abstract Number (optional), Pagination (optional).
     % Reference 7
     \bibitem[Author7(year)]{ref-thesis}
     Author 1, A.B. Title of Thesis. Level of Thesis, Degree-Granting University, Location of University, Date of Completion.
     % Reference 8
     \bibitem[Author8(year)]{ref-url}
     Title of Site. Available online: URL (accessed on Day Month Year).
 \end{thebibliography}


\usepackage{csquotes}

\usepackage{amsmath}
\usepackage{amsfonts}
\usepackage{amssymb}
\usepackage{amsthm}
\usepackage{bm}

\usepackage{graphicx}

\usepackage{authblk}

%\usepackage{natbib}
%\bibliographystyle{unsrt}
\usepackage{makeidx}
\usepackage[nottoc,notlot,notlof]{tocbibind}
%\usepackage{color}
%\usepackage{xcolor} % to access the named colour LightGray
%\definecolor{LightGray}{gray}{0.9}
\usepackage{ulem}

\usepackage{array}
\usepackage{booktabs}
\usepackage{multirow}
\usepackage{multicol}

\usepackage[shortlabels]{enumitem}

\usepackage[cache=false]{minted}
\usemintedstyle{friendly}
\renewcommand\listingscaption{Código}
%\usepackage{listings}
%\renewcommand{\lstlistingname}{Código}

\usepackage{float}
\usepackage{xparse}

\usepackage{siunitx}
\sisetup{group-minimum-digits=4}
\usepackage{physics}
\usepackage{tensor}
%\usepackage{derivative}
\usepackage[thinc]{esdiff}
\usepackage{stackengine}
\usepackage{mathtools}

\usepackage[backref=page, colorlinks=true, citecolor=cyan, linkcolor=blue, urlcolor=blue]{hyperref}
\usepackage[numbered]{bookmark}

%\usepackage{verbatim}
%\usepackage{attachfile}

glossaries-extra
fontspec
komascript
verse
pdflandscape

TexGyre fontspec
glossaries

tikz
beamer

tcolorbox

tcblisting?
ulem
Don't use ulem without \usepackage[normalem]{ulem}

microtype

float

tabularray
booktabs

Amsmath and its supplement math tools.

biblaetex

forest
systeme for typesetting systems of linear equations and inequalities.



multicol
fixme

auto multiple choice. 
lineno

ragged2e, bibtex, and natbib are my favorite packages. 
They are part of what convinced me that learning LaTeX was wort



\usepackage{textcomp}

\usepackage{fancyhdr}

\usepackage{import}

\usepackage{lipsum}


\usepackage{natbib}
\bibliographystyle{unsrt}
\usepackage{makeidx}
\usepackage[nottoc,notlot,notlof]{tocbibind}
\usepackage{color}
\usepackage{ulem}

\usepackage{array}
\usepackage{booktabs}
\usepackage{multirow}

\usepackage[shortlabels]{enumitem}

%\usepackage [framed, numbered, autolinebreaks]{mcode}
\usepackage{float}

\usepackage{xparse}
%\usepackage{physics}
\usepackage{siunitx}

\usepackage{geometry}


amssymb loads amsfonts. Instead of 
amsmath I use mathtools, which loads 
the former package. The latter one provides, 
for example, right-aligned matrix environments. 
babel is needed for other languages, while biblatex 
is usually used for citations and bibliography. T
he color package is called xcolor, and the graphics 
package is called graphicx.

I often use amsthm or ntheorem, which 

thermodynamics package

memoir class
