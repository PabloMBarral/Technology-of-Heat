\newpage
\section{Ligamentos}

Acá tengo que poner los casos simples y algunos de los casos complejos.

Ver cuando haga ese mini ejercicio para que entreguen, qué quiero poner, como para que tengan algo de variedad.



PG-52 estamos hablando.






\newpage
\section{Derivaciones y compensaciones}

PG-32

Como recomendación de diseño es dable pensar que no vamos a poner los cordones de soldadura, como hace Sobral.

Mandárselo para que le pegue una mirada.


\newpage
\section{Válvulas de seguridad}

PG-67.1
Minimum Number of Pressure Relief Valves Required

Dbe have two or more pressure relief valves

Los dispositivos de alivio de presión son las PSV, pressure safety valve, las válvulas de seguridad, coloquialmente.


PG-67.2
The total combined relieving capacity for
each boiler (except as noted in PG-67.2.1.6, PG-67.4,
and PL-54) shall be such that all the steam that can be
generated by the boiler is discharged without allowing
the pressure to rise more than 6\% above the highest

Acá se completa lo que vimos en la MAWP. No sólo la sobrepresión admisible cuando se están descargando, sino qué caudal deben descargar. Con esto, se pueden dimensionar los dispositivos.

PG-67.3
One or more pressure relief valves on the
boiler proper shall be set at or below the maximum allow-
able working pressure (except as noted in PG-67.4). If
additional valves are used the highest pressure setting
shall not exceed the maximum allowable working pres-
sure by more than 3\%. The complete range of pressure
settings of all the saturated-steam pressure relief valves
on a boiler shall not exceed 10\% of the highest pressure to
which any valve is set. Pressure setting of pressure relief
valves on high-temperature water boilers15 may exceed
this 10\% range. Economizer pressure relief devices
required by PG-67.2.1.6 shall be set as above using the
MAWP of the economizer


Poner una foto de una válvula de seguridad, que se vea el resorte ,el tornillo de calibracion, y una foto en la que se vean en una caldera, tanto en una humo como en el domo.

Chequear pero creo que tanto el eco como el sobrecalentador o el recalentador deben tener sus propios dispositivos de alivio.








\newpage
\section{Comentarios finales y conclusiones}

Poner aca todo lo que dejamos por afuera, o mencionar algunas de las partes.

Concluir algo.








%PAGINA 20 tabla 1A

%linea 45

%resistencia mecanica minima 485 MPa
%fluencia minima 260 MPa
%G10 S1 T2
%Limite maximo de temperartura 454 C

%buscar si tengo algun calculo ferrer de estos
%tiene soldaduras longitudinales y cirfunferenciales

%cabezal de medida distinta a la envueltas
%Ver PG-6, que me admite este material
%considerando que domo va aislado, lo vamos a poner todo a temperatura de saturacion

%apendice de grafitizacion

%asme II non mandatorio A 201 y 202

%mencionar algo de la eficiencia de la soldadura. Medio que pasarlo por alto, la verdad.

%Lo mismo sobre los casquetes, para no marearme yo y no marear a los estudianbtes.

%Sí poner lo de la grafitizacion.






