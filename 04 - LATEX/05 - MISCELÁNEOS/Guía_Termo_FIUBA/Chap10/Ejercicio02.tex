\section{Ejercicio}\label{ej:Chap10Ejercicio02}
\lipsum[1]


%En un ciclo Joule Brayton abierto, de aire e ideal, el aire ingresa al compresor a una
%presión de 100kPa y a una temperatira de 300 K, comprimiendo el aire ahasta 500kPa. La


%temperatura a la entrada de la turbina es de 900 K siendo la presión de descarga de la misma
%de 100 kP.
%Determinar:
%a) Temperatura a la salida del compresor y de la turbina.
%b) Calor intercambiado en kJ/kg
%c) Trabajo producido por la turbina en kJ/kg.
%d) Trabajo del compresor en kJ/kg
%e) Trabajo neto en kJ/kg
%f) Eficiencia térmica del ciclo.