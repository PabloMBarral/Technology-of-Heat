\section{Ejercicio}\label{ej:Chap10Ejercicio05}
\lipsum[1]


%Considerando que el ciclo del ejercicio anterior es un proceso real, y por lo tanto debe ser afectado por el rendimiento isoentrópico, siendo en la turbina ηTUR = 0,85 y en el
%compresor ηCOM = 0,8 . Datos: p1 = p4 = 100 kPa , t1 = 20°C , p2 =600 kPa , t3 = 827°C.
%%Determinar:
%a) Presión y temperatura en cada punto del ciclo.
%b) Trabajo en la turbina y en el compresor por cada kg de aire que circula.
%c) Rendimiento térmico del ciclo, ηt .
%d) Rendimiento exergético del ciclo, ηEX .
%e) Representar el ciclo en el mismo diagrama T-s .
%Suponer que los intercambiadores de calor con las fuentes son reversibles. Temperatura
%y presión del medio: : p0 = 100 kPa , T0 = 300 K .