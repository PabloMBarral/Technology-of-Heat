\section{Ejercicio}\label{ej:Chap04Ejercicio07}

Dos flujos, uno de aire atmosférico y el otro de agua, ambos en régimen estacionario, intercambian calor a través de una superficie. El aire entra a $\SI{32}{\celsius}$ y sale a $\SI{50}{\celsius}$, mientras que el agua (líquida) se enfría desde $\SI{88}{\celsius}$ hasta $\SI{60}{\celsius}$, y su caudal es de $\SI{2100}{\frac{l}{h}}$.

Determinar el caudal de aire necesario (y expresarlo en $\SI{}{\frac{m^3}{min}}$) y el calor intercambiado entre ambas masas en circulación.