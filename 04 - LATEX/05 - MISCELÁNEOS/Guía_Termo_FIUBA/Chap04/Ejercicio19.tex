\section{Ejercicio}\label{ej:Chap04Ejercicio19}

El motor de un automóvil entrega una potencia máxima que es proporcional a la masa de aire que aspira. Para aumentarla, se recurre a comprimir el aire de entrada adiabáticamente hasta $\SI{0.22}{MPa(a)}$, para que ocupe menos volumen. Durante la compresión, la temperatura aumenta hasta $\SI{140}{\celsius}$. Luego, se lo enfría hasta $\SI{60}{\celsius}$, con la misma finalidad, y perdiendo $\SI{0.01}{MPa(a)}$ por fricción en este último proceso.

Calcular el aumento de potencia máxima del motor con el compresor solamente, y, por otro lado, con el compresor y el enfriador, respecto del motor de aspiración natural.

Las condiciones atmosféricas del aire de aspiración son $\SI{0.1013}{MPa(a)}$ y $\SI{30}{\celsius}$.

Representar en un diagrama $p-v$.