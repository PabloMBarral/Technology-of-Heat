\section{Ejercicio}\label{ej:Chap03Ejercicio03}
En un tanque se encuentra una mezcla de gases en equilibrio, la cual está compuesta por $21\%$ de oxígeno y $79\%$ de nitrógeno, en volumen. La temperatura es de $\SI{20}{\celsius}$, el volumen total es de $\SI{50}{m^3}$ y la presión total es de $\SI{89}{kPa(a)}$.

Determinar, considerando a ambos gases como ideales
\begin{enumerate}
    \item La presión parcial de cada uno de los gases.
    \item La masa de cada uno.
    \item El $R_M$ de la mezcla.
\end{enumerate}
Si esta misma mezcla de gases de expandiera hasta ocupar el doble del volumen original, y lo hiciera a la misma temperatura, calcular la presión total utilizando la ecuación de estado de los gases ideales, con la constante de la mezcla $R_M$.