\documentclass[10pt,a4paper,oneside]{book}
\usepackage[utf8]{inputenc}
\usepackage[spanish]{babel}
\usepackage[T1]{fontenc}
\usepackage{lmodern}
\usepackage{microtype}

\usepackage{amsmath}
\usepackage{amsfonts}
\usepackage{amssymb}
\usepackage{amsthm}
\usepackage{bm}

\usepackage{graphicx}

\usepackage{authblk}

%\usepackage{natbib}
%\bibliographystyle{unsrt}
\usepackage{makeidx}
\usepackage[nottoc,notlot,notlof]{tocbibind}
%\usepackage{color}
%\usepackage{xcolor} % to access the named colour LightGray
%\definecolor{LightGray}{gray}{0.9}
\usepackage{ulem}

\usepackage{array}
\usepackage{booktabs}
\usepackage{multirow}
\usepackage{multicol}

\usepackage[shortlabels]{enumitem}

\usepackage[cache=false]{minted}
\usemintedstyle{friendly}
\renewcommand\listingscaption{Código}
%\usepackage{listings}
%\renewcommand{\lstlistingname}{Código}

\usepackage{float}
\usepackage{xparse}

\usepackage{siunitx}
\sisetup{group-minimum-digits=4}
\usepackage{physics}
\usepackage{tensor}
%\usepackage{derivative}
\usepackage[thinc]{esdiff}
\usepackage{stackengine}
\usepackage{mathtools}

\usepackage[backref=page, colorlinks=true, citecolor=cyan, linkcolor=blue, urlcolor=blue]{hyperref}
\usepackage[numbered]{bookmark}

%\usepackage{verbatim}
%\usepackage{attachfile}

\usepackage{textcomp}

\usepackage{fancyhdr}

\usepackage{import}

\usepackage{lipsum}

\title{Guía de ejercicios}
\author{67.04 Termodinámica IA}
\affil{Depto. de Ing. Mecánica - FIUBA}
\affil{Prof. Ing. Marcelo Turchetti\thanks{ \,\texttt{marcelo.turchetti@gmail.com}}}
\affil{Ing. Pablo Barral\thanks{\,\texttt{pbarral@fi.uba.ar}}}

\date{\today \\Revisión 0}

%Realizó: PMB - 08-11-2021
%Mejoras:
%1)

\begin{document}
\pagestyle{fancy}
\fancyhf{}
\lhead{\chaptername \,\,\thechapter}
\lfoot{Marcelo Turchetti - Pablo Barral}
\rfoot{Página \thepage}

\maketitle
\tableofcontents

%%%%%%%%%%%%%%%%%%%%%%%%%%%%%%%%%%%%%%%%%%%%%%%%%%%%%%%%%%%%%%%%%%%%
\chapter{Conceptos fundamentales y unidades}
\import{Chap01}{Ejercicio01}
\import{Chap01}{Ejercicio02}
\import{Chap01}{Ejercicio03}
\import{Chap01}{Ejercicio04}
\import{Chap01}{Ejercicio05}
\import{Chap01}{Ejercicio06}
\import{Chap01}{Ejercicio07}
\import{Chap01}{Ejercicio08}
%%%%%%%%%%%%%%%%%%%%%%%%%%%%%%%%%%%%%%%%%%%%%%%%%%%%%%%%%%%%%%%%%%%%
\chapter{Sistemas termodinámicos}
\import{Chap02}{Ejercicio01}
\import{Chap02}{Ejercicio02}
\import{Chap02}{Ejercicio03}
%%%%%%%%%%%%%%%%%%%%%%%%%%%%%%%%%%%%%%%%%%%%%%%%%%%%%%%%%%%%%%%%%%%%
\chapter{Estados termodinámicos de gases ideales, mezclas, y gases
reales}
\import{Chap03}{Ejercicio01}
\import{Chap03}{Ejercicio02}
\import{Chap03}{Ejercicio03}
\import{Chap03}{Ejercicio04}
\import{Chap03}{Ejercicio05}
\import{Chap03}{Ejercicio06}
\import{Chap03}{Ejercicio07}
\import{Chap03}{Ejercicio08}
\import{Chap03}{Ejercicio09}
\import{Chap03}{Ejercicio10}
\import{Chap03}{Ejercicio11}
\import{Chap03}{Ejercicio12}
%%%%%%%%%%%%%%%%%%%%%%%%%%%%%%%%%%%%%%%%%%%%%%%%%%%%%%%%%%%%%%%%%%%%
\chapter{Primer principio de la termodinámica}
\import{Chap04}{Ejercicio01}
\import{Chap04}{Ejercicio02}
\import{Chap04}{Ejercicio03}
\import{Chap04}{Ejercicio04}
\import{Chap04}{Ejercicio05}
\import{Chap04}{Ejercicio06}
\import{Chap04}{Ejercicio07}
\import{Chap04}{Ejercicio08}
\import{Chap04}{Ejercicio09}
\import{Chap04}{Ejercicio10}
\import{Chap04}{Ejercicio11}
\import{Chap04}{Ejercicio12}
\import{Chap04}{Ejercicio13}
\import{Chap04}{Ejercicio14}
\import{Chap04}{Ejercicio15}
\import{Chap04}{Ejercicio16}
\import{Chap04}{Ejercicio17}
\import{Chap04}{Ejercicio18}
\import{Chap04}{Ejercicio19}
\import{Chap04}{Ejercicio20}
\import{Chap04}{Ejercicio21}
\import{Chap04}{Ejercicio22}
\import{Chap04}{Ejercicio23}
\import{Chap04}{Ejercicio24}
\import{Chap04}{Ejercicio25}
\import{Chap04}{Ejercicio26}
\import{Chap04}{Ejercicio27}
\import{Chap04}{Ejercicio28}
%%%%%%%%%%%%%%%%%%%%%%%%%%%%%%%%%%%%%%%%%%%%%%%%%%%%%%%%%%%%%%%%%%%%
\chapter{Transformaciones politrópicas}
\import{Chap05}{Ejercicio01}
\import{Chap05}{Ejercicio02}
\import{Chap05}{Ejercicio03}
\import{Chap05}{Ejercicio04}
\import{Chap05}{Ejercicio05}
\import{Chap05}{Ejercicio06}
\import{Chap05}{Ejercicio07}
%%%%%%%%%%%%%%%%%%%%%%%%%%%%%%%%%%%%%%%%%%%%%%%%%%%%%%%%%%%%%%%%%%%%
\chapter{Compresores alternativos de pistón}
\import{Chap06}{Ejercicio01}
\import{Chap06}{Ejercicio02}
%%%%%%%%%%%%%%%%%%%%%%%%%%%%%%%%%%%%%%%%%%%%%%%%%%%%%%%%%%%%%%%%%%%%
\chapter{Segundo principio de la termodinámica y entropía}
\import{Chap07}{Ejercicio01}
\import{Chap07}{Ejercicio02}
\import{Chap07}{Ejercicio03}
\import{Chap07}{Ejercicio04}
\import{Chap07}{Ejercicio05}
\import{Chap07}{Ejercicio06}
\import{Chap07}{Ejercicio07}
\import{Chap07}{Ejercicio08}
\import{Chap07}{Ejercicio09}
\import{Chap07}{Ejercicio10}
\import{Chap07}{Ejercicio11}
\import{Chap07}{Ejercicio12}
\import{Chap07}{Ejercicio13}
\import{Chap07}{Ejercicio14}
\import{Chap07}{Ejercicio15}
\import{Chap07}{Ejercicio16}
\import{Chap07}{Ejercicio17}
\import{Chap07}{Ejercicio18}
\import{Chap07}{Ejercicio19}
\import{Chap07}{Ejercicio20}
\import{Chap07}{Ejercicio21}
\import{Chap07}{Ejercicio22}
\import{Chap07}{Ejercicio23}
\import{Chap07}{Ejercicio24}
\import{Chap07}{Ejercicio25}
\import{Chap07}{Ejercicio26}
\import{Chap07}{Ejercicio27}
\import{Chap07}{Ejercicio28}
\import{Chap07}{Ejercicio29}
%%%%%%%%%%%%%%%%%%%%%%%%%%%%%%%%%%%%%%%%%%%%%%%%%%%%%%%%%%%%%%%%%%%%
\chapter{Exergía}
\import{Chap08}{Ejercicio01}
\import{Chap08}{Ejercicio02}
\import{Chap08}{Ejercicio03}
\import{Chap08}{Ejercicio04}
\import{Chap08}{Ejercicio05}
\import{Chap08}{Ejercicio06}
\import{Chap08}{Ejercicio07}
\import{Chap08}{Ejercicio08}
\import{Chap08}{Ejercicio09}
\import{Chap08}{Ejercicio10}
\import{Chap08}{Ejercicio11}
\import{Chap08}{Ejercicio12}
\import{Chap08}{Ejercicio13}
\import{Chap08}{Ejercicio14}
%%%%%%%%%%%%%%%%%%%%%%%%%%%%%%%%%%%%%%%%%%%%%%%%%%%%%%%%%%%%%%%%%%%%
\chapter{Sustancias puras}
\import{Chap09}{Ejercicio01}
\import{Chap09}{Ejercicio02}
\import{Chap09}{Ejercicio03}
\import{Chap09}{Ejercicio04}
\import{Chap09}{Ejercicio05}
\import{Chap09}{Ejercicio06}
\import{Chap09}{Ejercicio07}
\import{Chap09}{Ejercicio08}
\import{Chap09}{Ejercicio09}
\import{Chap09}{Ejercicio10}
\import{Chap09}{Ejercicio11}
\import{Chap09}{Ejercicio12}
\import{Chap09}{Ejercicio13}
\import{Chap09}{Ejercicio14}
\import{Chap09}{Ejercicio15}
\import{Chap09}{Ejercicio16}
\import{Chap09}{Ejercicio17}
%%%%%%%%%%%%%%%%%%%%%%%%%%%%%%%%%%%%%%%%%%%%%%%%%%%%%%%%%%%%%%%%%%%%
\chapter{Ciclos de potencia}
\import{Chap10}{Ejercicio01}
\import{Chap10}{Ejercicio02}
\import{Chap10}{Ejercicio03}
\import{Chap10}{Ejercicio04}
\import{Chap10}{Ejercicio05}
\import{Chap10}{Ejercicio06}
\import{Chap10}{Ejercicio07}
\import{Chap10}{Ejercicio08}
\import{Chap10}{Ejercicio09}
\import{Chap10}{Ejercicio10}
\import{Chap10}{Ejercicio11}
\import{Chap10}{Ejercicio12}
\import{Chap10}{Ejercicio13}
%%%%%%%%%%%%%%%%%%%%%%%%%%%%%%%%%%%%%%%%%%%%%%%%%%%%%%%%%%%%%%%%%%%%
\chapter{Ciclos frigoríficos}
\import{Chap11}{Ejercicio01}
\import{Chap11}{Ejercicio02}
\import{Chap11}{Ejercicio03}
\import{Chap11}{Ejercicio04}
\import{Chap11}{Ejercicio05}
\import{Chap11}{Ejercicio06}
\import{Chap11}{Ejercicio07}
\import{Chap11}{Ejercicio08}
\import{Chap11}{Ejercicio09}
\import{Chap11}{Ejercicio10}
\import{Chap11}{Ejercicio11}
\import{Chap11}{Ejercicio12}
\import{Chap11}{Ejercicio13}
\import{Chap11}{Ejercicio14}
\import{Chap11}{Ejercicio15}
\import{Chap11}{Ejercicio16}
\import{Chap11}{Ejercicio17}
\import{Chap11}{Ejercicio18}
%%%%%%%%%%%%%%%%%%%%%%%%%%%%%%%%%%%%%%%%%%%%%%%%%%%%%%%%%%%%%%%%%%%%
\chapter{Aire húmedo}
\import{Chap12}{Ejercicio01}
\import{Chap12}{Ejercicio02}
\import{Chap12}{Ejercicio03}
\import{Chap12}{Ejercicio04}
\import{Chap12}{Ejercicio05}
\import{Chap12}{Ejercicio06}
\import{Chap12}{Ejercicio07}
\import{Chap12}{Ejercicio08}
\import{Chap12}{Ejercicio09}
\import{Chap12}{Ejercicio10}
\import{Chap12}{Ejercicio11}
\import{Chap12}{Ejercicio12}
\import{Chap12}{Ejercicio13}
\import{Chap12}{Ejercicio14}
\import{Chap12}{Ejercicio15}
\import{Chap12}{Ejercicio16}
\import{Chap12}{Ejercicio17}
%%%%%%%%%%%%%%%%%%%%%%%%%%%%%%%%%%%%%%%%%%%%%%%%%%%%%%%%%%%%%%%%%%%%
\chapter{Termoquímica y combustión}
\import{Chap13}{Ejercicio01}
\import{Chap13}{Ejercicio02}
%%%%%%%%%%%%%%%%%%%%%%%%%%%%%%%%%%%%%%%%%%%%%%%%%%%%%%%%%%%%%%%%%%%%
\chapter{Respuestas}
%%%%%%%%%%%%%%%%%%%%%%%%%%%%%%%%%%%%%%%%%%%%%%%%%%%%%%%%%%%%%%%%%%%%
\chapter{Resueltos}
\end{document}

% ojo que asi como estan los marcadores no me sirven porque dicen solo "Ejercicio". No quise poner el numero para no repetir el tag. Pero me gustaria que en el marcador se vea. Lo que no quiero poner es 11.2. Ejercicio 11-2. porque es redundante y feo.

%en las respuestas para el que aproximo el calor especifico poner la regresion lineal que hago con EES y el graico que obtengo de correlacion perfecta (recta diagonal) y de la aproximacin. Y grficar ambas funciones, tambien.

% hacer eso con las respuestas, trabajarslas.

% los diagramas ponerlos en el apendice
% robar buenos graficos de los libros, ver con geometry cambiar los margenes, ver la orientacion, capaz
%poner aplicaciones, UTAs en aire humedo, por ejemplo, hablar de chillers

%comprimir metano, darle un matiz mas bien ingenieril a todo

% en el de exergia ponerle que repitan alguno de los de primer principio, los mas relevantes

% embeberle ejecutables para que jueguen y prueben cosas

% Ponerle links a videos de mi drive, y a clases grabadas
% Revisar los números de las soluciones
% Poner esquema de resolución.
% Ponerle un comentario de la bibliografía detrás
% Poner ejercicios en Python
% Ponerle herramientas que les sean útiles a ellos.

% Luego hacer un apunte de la cátedra, con las cosas que vamos viendo en la práctica,
% de a poco ir mejorando las imagenes
% agregar los parciales que fui resolviendo.

% Primero poner la guía tal cual, corrigiendo estilo.
% Luego leer lo que pongo a ver si tiene sentido

% Hacerlo bien interactivo para ir y volver en el pdf

% incluir los archivos EES acá en la misma guía. Y que estos archivos estén en forma zip, impresos, y bien escrito lo que resuelvo
% y que digan todo lo que tienen incluido, si tienen tablas parametricas, graficos, etc.

% El apéndice de fórmulas y resumen, ponerlos en pdfs aparte.

% Incluir el de Usain Bolt y eso.
% Ver de hacer alguna comparación entre las magnitudes


%Notas:
%1) Poner abatimiento de condensado, atemperación y tanque flash.
%2) Poner alguna explicación sobre un sistema VRV
%3) Compresión de aire húmedo, drenaje de tanque
%4) Ir mejorando los dibujos
%5) Balance térmico de calor sensible y latente aire húmedo
%6) Alguno de termoquímica y combustión
%7) Poner los que doy en clase
%8) Poner el choice que ya tomé
%9) Poner el que comprime oxígeno en la succión.
%10) Tener en cuenta que son 10 guías y 16 semanas. No me puedo zarpar, tampoco.
%11) Poner alguno con entalpía de remanso.
%12) Usar la guía para introducir conceptos.
%13) Hacer al menos uno de cada tipo resuelto.
%14) Poner algo interesante sobre la performance de los compreores alternativos a piston.
%15) Fomentar el uso de python y de ees. Dejar links a las librerías. Incluir capturas de pantalla, o ver de hacer algún powerpoint.
%16) Torre de enfriamiento
%17) Hacer el de la turbina de gas que hice.
%18) ver ese de calor maximo y minimo.
%19) En la de sustancias puras poner  algun con refrigerante, tambien
%20) Poner algun ciclo de turbina de gas con recalentamiento, y con calentamiento regeenerativo.
%21) Poner alguno de un ciclo combinado, ver cómo darles la parte de la caldera.
%22) Algo con diámetros de ductos.
%23) Membranas semipermeables
%24) Tratar de que haya humor en los ejercicios.
%25) Tratar de incluir ejercicios resueltos
%26) Tratar de compendiar los archivos EES que fui haciendo.Colgarlo del campus.
%27) Régimen de aprobación hacer. Hacer prototipo de cronograma.
%28) poner el de yacyreta y el agua para el mate, el de usain bolt y la calefaccion del ambiente.
%29) En la parte de exergia citar a los ejercicios de la primea parte, tambien.
%30) tratar de meter algo de entropia en combustion y en aire humedo.
%31) No quiero tanto truquito sino que sepan calcular cosas, que las conozcan, onda informativo. y que entiendan los conceptos. 
%32) Por lo tanto, que los problemas tengan cosas para pensar.
%33) Revisar todo lo que haya puesto y mejorar los dibujos, agregar cosas que pido.
%34) Mejorar los ejercicios que hay en la guía de Turchetti.
%35) Tratar que los ejercicios que pongo con situaciones incluyan conceptos. Y que los conceptos vayan gradualmente. Pensarlo bien esto.
%36) Deberían ser 10 por tema. 100 ejercicios.
%37) Rehacer el apéndice de fórmulas tambien.
%38) Tener en cuenta que hay pelotudos que hacen todos los ejercicios sin pensar mucho.
%39) hacer una guia de estudio mostrando cosas que quiero que vean.
%40) poner imagenes ilustrativas de los compresores, las torres de enfriamiento, los intercambiadores de calor, las calderas, turbinas, toberas, usain bolt, yacyreta, el termo de agua. La idea es que estén motivados para hacer los ejercicios.
%41) Los resultados, al final. Tratar de que haya hipervinculos. Y resueltos, pero eso con moderacion, que es bocha de trabajo. 
%43) Que haya alguno por equipo: uno de tobera, uno de HX, uno de cmara de mezcla
%44) El de la camara de mezcla que muestre cuándo se puede y cuándo no se puede sumar volumenes.


%Están pensados para que les lleven 1h 1,5hs cada uno. Y tienen que dedicarle, además de las clases, lo mismo en sus casas. Así que serían 64 hs en el cuatrimestre. Tratar de repartir 32 y 32 entre la primera parte y la segunda. Ver cuantas clases son.

%que haya precalentadores de agua y tanque de agua de alimentacion
%alguno que diga para usar software, y hacer alguna sensibilidad
%mencionar algun vrv
%algun balance temrico hvac
%compresor de aire humedo y recolectar condensado
%algun proceso en abstracto
%algo de secado


%Tanque flash
%El que hacemos siempre de vaporizacion a volumen constante
%Desobrecalentamiento
%Abatimiento de condensado


% Algo interesante que sucede es que no lo que no hago en clase, no tengo forma de saber si lo hicieron o no. En principio, es dado asumir que no lo miran. Por ejemplo, entropía en sistemas transitorios.

% Voy a tener que reformular la forma en que doy clase. Armarme unos ppts que me sirvan de guía y de apoyo al pizarron.


%\begin{table}[ht!]
%\centering
%\begin{tabular}[t]{cccccc}
%\toprule
%$\mathbf{m_{aire}\,[kg]}$ &$\mathbf{p_f\,[kPa]}$ &$\mathbf{t_f\,[^{\circ}C]}$ &$\mathbf{Q_{tot}}\,[kJ]$ &$\mathbf{W_{tot}\,[kJ]}$&\textbf{¿Es cuasiestática? (S/N) } \\
%\midrule
% & & & & & \\
%\midrule
%$\mathbf{\Delta S_{aire}\,[kJ/K]}$ &$\mathbf{\Delta S_u\,[kJ/K]}$ &$\mathbf{\Delta E_{x,\,aire}\,[kJ]}$ &$\mathbf{\Delta E_{x,\,u}\,[kJ]}$ &$\mathbf{\eta_{ex}\,[\%]}$ &\textbf{¿Es reversible? (S/N) }\\
%\midrule
% & & & & & \\
%\bottomrule
%\end{tabular}
%\caption{\textit{Completar con resultados.}}
%\label{tab:Resultado}
%\end{table}


