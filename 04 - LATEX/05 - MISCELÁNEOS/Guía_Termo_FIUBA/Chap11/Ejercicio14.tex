\section{Ejercicio}\label{ej:Chap11Ejercicio14}
En un ciclo frigorífico de doble compresión se extrae en el evaporador una
cantidad de calor de $\SI{209}{\frac{MJ}{h}}$ a $\SI{-20}{\celsius}$, condensando a $\SI{40}{\celsius}$.

La temperatura de aspiración de baja se adopta en $\SI{-10}{\celsius}$. Para la presión intermedia, adoptar $p_i=\sqrt{p_e \cdot p_c}$. El compresor de alta aspira vapor saturado. El fluido de trabajo es Freon 12 (R-12).

Determinar
\begin{enumerate}
    \item Caudal másico de Freon.
    \item Potencia consumida.
    \item Calor cedido en el condensador.
    \item COP.
\end{enumerate}