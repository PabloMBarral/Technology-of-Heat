\section{Ejercicio}\label{ej:Chap07Ejercicio29}
Dentro de un cilindro cerrado por un pistón sin rozamiento hay inicialmente una masa de aire a una temperatura $t_i=\SI{50}{\celsius}$ y a una presión de $\SI{1000}{kPa(a)}$,

Ingresan dos maseas de aire, una $m_1=\SI{10}{\kg}$ a $t_1=\SI{20}{\celsius}$ y otra $m_2=\SI{3}{\kg}$ a $t_2=\SI{35}{\celsius}$. Además, sale otra masa $m_3=\SI{12}{kg}$ a una temperatura desconocida.

Se entrega calor al cilindro de una fuente térmica a una temperatura de $T_H=\SI{1000}{\kelvin}$. El volumen inicial del cilindro es de $\SI{1}{m^3}$ y el final es del doble. La temperatura de salida es igual a la temperatura final $t_f$ dentro del cilindro.

Calcular
\begin{enumerate}
    \item La temperatura final de la masa de aire que queda dentro del cilindro.
    \item La cantidad de calor suministrada por la fuente térmica.
    \item La variación de entropía del aire y la del universo.
    \item Decir si el proceso fue reversible o no. Justificar, evidenciando los procesos irreversibles o reversibles.
\end{enumerate}