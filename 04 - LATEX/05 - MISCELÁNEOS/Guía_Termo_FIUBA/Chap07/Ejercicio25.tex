\section{Ejercicio}\label{ej:Chap07Ejercicio25}
Una máquina térmica con un rendimiento de $\eta_T=0.8 \times \eta_c$, siendo $\eta_c$ el rendimiento de la máquina de \textit{Carnot}, funciona entre $2$ fuentes de calor, cuyas temperaturas son $T_{f_1}=\SI{1000}{\kelvin}$ y $T_{f_2}=\SI{400}{\kelvin}$. Esta máquina entrega un $W_{paletas}=\SI{2130}{kJ}$ a un cilindro adiabático que contiene aire en las condiciones iniciales $t_i=\SI{27}{\celsius}$ y $V_i=\SI{5}{m^3}$.

Si el cilindro tiene un pistón que no posee peso, determinar el calor consumido de la fuente caliente.