\section{Ejercicio}\label{ej:Chap12Ejercicio02}
En un ambiente el termómetro marca $t_{bs}=\SI{25}{\celsius}$ y un termómetro de bulbo húmedo marca $t_{bh}=\SI{20}{\celsius}$. La presión total es de $p_{tot}=\SI{760}{mmHg}$.

Ubicar el estado termodinámico en un diagrama psicrométrico y en un diagrama de \textit{Mollier}. Obtener de allí el resto de los parámetros (entalpía, humedad absoluta y relativa, presión parcial de vapor, temperatura de rocío, calor específico a presión constante del aire húmedo, etc.).

Además, determinar los mismos parámetros mediante expresiones analíticas.