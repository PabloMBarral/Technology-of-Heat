\section{Ejercicio}\label{ej:Chap12Ejercicio09}
Aire húmedo a $\SI{17}{\celsius}$, $\SI{760}{mmHg}$ y humedad relativa $\varphi_1=40\%$ debe ser llevado, a presión constante, hasta $\SI{28}{\celsius}$ y $\varphi_2=60\%$ mediante los siguientes procesos, en el orden dado:
\begin{enumerate}
    \item Calentamiento.
    \item Humidificación con agua líquida hasta saturación.
    \item Calentamiento.
\end{enumerate}

Se pide
\begin{enumerate} [a)]
    \item Determinar las condiciones del aire después de cada proceso, de manera gráfica y de manera analítica.
    \item Si el mismo estado final se alcanza sólo con el agregado de vapor de agua a $\SI{10}{bar(a)}$, determinar el estado del vapor y la cantidad a agregar.
\end{enumerate}