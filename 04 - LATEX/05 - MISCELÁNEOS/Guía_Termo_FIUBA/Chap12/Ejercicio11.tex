\section{Ejercicio}\label{ej:Chap12Ejercicio11}
Una masa $\dot{m}_{1(a.h.)}=\SI{1000}{\frac{kg}{h}}$ de aire húmedo, que se encuentra a $p_{1(T)}=\SI{101.325}{kPa(a)}$. $T_{1(b.s.)}=\SI{20}{\celsius}$


Calcular:
\begin{enumerate}
    \item cantidad de aire seco que componen las masas de aire húmedo $\dot{m}_1$ y $\dot{m}_2$.
    \item humedad relativa del estado $2$ $\varphi_2$
    \item temperatura $t_3$ a la salida de la cámara de mezcla.
    \item humedad relativa del estado $3$ $\varphi_3$.
    \item estado del agua para la humidificación.
    \item cantidad de agua agregada $\dot{w}$ en $\SI{}{\frac{kg}{h}}$.
\end{enumerate}

NOTA: Resolver el problema analíticamente y gráficamente empleando el diagrama de Mollier.