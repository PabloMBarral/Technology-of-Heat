\section{Ejercicio}\label{ej:Chap08Ejercicio03}

% TERMINAR

Para el mismo estado de referencia del ejercicio anterior, $T_0=\SI{300}{K}$ y $p_0=\SI{1}{bar(a)}$, calcular las variaciones de exergía de los siguientes casos:
\begin{enumerate}
    \item $\SI{1}{kg}$ de aire (gas ideal) a $t_1=\SI{650}{\celsius}$ y $p_1=\SI{12}{bar(a)}$ se expande isoentrópicamente hasta la presión del medio de referencia.
    \item Una bala de cañón de $\SI{10}{kg}$, que se desplaza horizontalmente $\SI{250}{\frac{m}{s}}$, baja su velocidad a $\SI{150}{\frac{m}{s}}$ debido al rozamiento con el aire.
    \item Una corriente de 300 kg/h de aire a p1=1,2 kgf/cm2 y T1=300K se calientan a una presión constante hasta T2=500K.
    \item Un recipiente 1m3 de capacidad donde se ha realizado vacío inicialmente, debido a fugas, es llenado por el aire ambiente, quedando en equilibrio con el medio de referencia.
    \item Una fuente térmica de capacidad calorífica infinita a $T_F=\SI{1200}{K}$ recibe $\SI{1000}{kcal}$.
    \item A una tobera adiabática reversible ingresan $\SI{100}{\frac{kg}{h}}$ de aire a baja velocidad y unas condiciones $p_1=\SI{10}{\frac{kg_f}{cm^2}}$ y $t_1=\SI{80}{\celsius}$, y sale a $\SI{200}{\frac{m}{s}}$, con una presión de $p_2=\SI{8}{\frac{kg_f}{cm^2}}$.
\end{enumerate}