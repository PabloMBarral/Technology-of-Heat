\section{Ejercicio}\label{ej:Chap06Ejercicio01}

Se desea construir un compresor alternativo para comprimir aire desde $p_1=\SI{1}{bar(a)}$ y $\SI{20}{\celsius}$ hasta una presión de $p_2=\SI{9}{bar(a)}$. El compresor debe funcionar a $\SI{350}{rpm}$ (revoluciones por minuto. Se planea con una relación de espacio nocivo igual al $2\%$, y se supone que el exponente de la politrópica es $m=1.25$, siendo la relación entre la carrera $L$ y el díametro $D$ del pistón $\beta=\frac{L}{D}=1.2$. El gasto másico es de $\SI{300}{\frac{\kg}{h}}$.

Determinar:
\begin{enumerate}
    \item Trabajo a suministrar por cada unidad de masa que circula, trabajo total por hora. Potencia requerida en $\SI{}{kW}$ y dimensiones principales: diámetro y carrera.
    \item Cantidad de calor que debe transferir al exterior por hora.
    \item Si el compresor diseñado se utiliza para comprimir aire hasta $\SI{18}{bar(a)}$, ¿cómo se modifica el gasto másico y la potencia de accionamiento?
    \item Si se construyea un compresor de dos etapas y enfriamiento intermedio para cumplir con los requerimientos enunciados, ¿cuál sería la potencia total de accionamiento en estas circunstancias?
    \item Si el compresor de dos etapas ya diseñado se empleara para llenar un tanque a la presión de $\SI{18}{bar(a)}$, ¿cuál sería el gasto másico que podría circular y la potencia del motor de accionamiento?
\end{enumerate}
