\maketitle

\begin{abstract}
     Esto es el resumen del texto.
\end{abstract}






% ############################

% GUÍA DE TEMAS

% Rendimiento térmico y exergético, postcombustión, diagramas de flujo, qué pasa con las presiones con la posctombustión
% pinch y approach
% reglas para carga parcial
% integración del DA
% HX de placas a la entrada
% balance de calderas de varias presiones
% ecuaciones de balance
% pérdida al ambiente
% balance en una TG
% limites de la postcombustión
% introducción a las ecuaciones de dimensionamiento (ESCOA)
% pérdida de carga
% usar cp o caracteristica de los gases
% limites en la chimenea
% diagrama TQ y -1TvsQ
% Variedad de diagramas de flujo, formas de armarla
% Analisis ese de una presion, dos, tres, recalentador sí o no.

% Explicar que pinch y approach se especifican, luego sacamos cantidad de vapor.
% Poner qué es dato y qué resultado.
% Ver libros específicos, copiar a Ganapathy.




















% ############################

%\vspace{1em}
%\noindent{\small\textbf{Palabras clave:} ASME, BPVC, calderas.}

%\begin{quote}
%     \fbox{%
%     \parbox{\dimexpr\linewidth-2\fboxsep}{%
%         \textbf{Disclaimer:} El contenido de este apunte tiene como única finalidad ser una introducción somera al código en el marco de una asignatura de grado, apoyándose en una exposición oral. Este apunte no constituye, bajo ningún tipo de concepto, un reemplazo del código, ni en su versión 2023 ni en la última vigente. Este apunte no sustituye de ninguna manera al juicio criterioso en el diseño ni a las reglas del buen arte y la experiencia en el diseño, la fabricación, el ensayo, la inspección y la operación. El autor no asume responsabilidad por ninguna acción tomada basada en la información proporcionada en este apunte, y prohíbe enfáticamente el uso de este apunte para el diseño. Finalmente, se recuerda que el diseño de recipientes a presión sin una validación empírica o siguiendo los lineamientos de un ente reconocido constituye un serio peligro para la seguridad, la integridad de los equipos y las instalaciones y la vida de las personas, con consecuencias posiblemente fatales o incapacitantes de modo permanente.
%     }
%     }
%\end{quote}


