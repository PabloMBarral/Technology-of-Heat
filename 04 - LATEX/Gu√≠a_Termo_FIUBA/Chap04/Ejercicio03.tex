\section{Ejercicio}\label{ej:Chap04Ejercicio03}
A un tanque rígido y adiabático formado por una mezcla de sustancias desconocidas se lo agita mediante paletas movidas por un motor eléctrico, y simultáneamente se lo calienta con una resistencia eléctrica inmersa en la mezcla.

El proceso dura $20$ minutos. El motor consume $\SI{1.5}{kW}$ (supuesto sin pérdidas, o sea que produce $\SI{1.5}{kW}$ de energía mecánica); la resistencia disipa una potencia de $\SI{3.3}{\kilo\watt}$. Aplicando el primer principio de la termodinámica, determinar la variación de energía interna del sistema formado por el contenido del tanque adiabático.