\section{Ejercicio}\label{ej:Chap10Ejercicio04}
\lipsum[1]


%En un ciclo Brayton de aire estándar, el gas ingresa al compresor a p1 = 100 kPa y
%t1 = 20°C. La presión a la salida del compresor es p2 =600 kPa y la temperatura máxima del
%ciclo es de t3 = 827°C. Suponiendo procesos ideales, calcular:
%a) Presión y temperatura en cada punto del ciclo.
%b) Trabajo en la turbina y en el compresor por cada kg de aire que circula.
%c) Rendimiento térmico del ciclo. Hallar la expresión del rendimiento térmico en
%función de la relación p2 / p1 