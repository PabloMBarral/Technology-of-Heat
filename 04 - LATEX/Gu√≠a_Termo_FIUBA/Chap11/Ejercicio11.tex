\section{Ejercicio}\label{ej:Chap11Ejercicio11}
Un ciclo a doble compresión con enfriamiento intermedio extrae una cantidad de calor del evaporador de $\SI{50000}{\frac{kcal}{h}}$ a una temperatura de $\SI{-30}{\celsius}$, bajo las siguientes condiciones
\begin{itemize}
    \item Temperatura de condensación: $\SI{40}{\celsius}$.
    \item Fluido: \textit{Freón} $22$.
    \item Temperatura de aspiración en el compresor de baja presión: $\SI{-10}{\celsius}$.
    \item Temperatura de aspiración en el compresor de alta presión: $\SI{10}{\celsius}$.
    \item Presión intermedia: $p_i=\sqrt{p_e\,p_c}$, siendo $p_e$ la presión del evaporador y $p_c$ la presión del condensador.
    \item Salida del evaporador: vapor saturado (evaporador inundado).
    \item Salida del condensador: líquido saturado.
\end{itemize}

Determinar
\begin{enumerate}
    \item Masa horaria de \textit{Freón} que circula.
    \item Potencia absorbida en la compresión.
    \item Calor por extraer en el condensador.
    \item Coeficiente de performance (COP).
\end{enumerate}