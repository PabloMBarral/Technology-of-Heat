\section{Ejercicio}\label{ej:Chap11Ejercicio18}

Un avión de pasajeros que está próximo a despegar es refrigerado en su interior mediante un ciclo abierto de refrigeración de gas que opera con aire. El compresor aspira aire atmosférico a $\SI{303}{\kelvin}$ y $\SI{100}{kPa(a)}$ y lo comprime hasta $\SI{200}{kPa(a)}$. 

El aire pasa por un intercambiador de calor y disminuye su temperatura hasta $\SI{353}{\kelvin}$. La turbina descarga a la presión atmosférica en el interior del avión. El rendimiento isoentrópico del compresor y la turbina son iguales a $0.95$.

Determinar la temperatura del aire en la cabina del avión y la potencia que consume por unidad de masa de aire.