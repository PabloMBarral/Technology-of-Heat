\section{Ejercicio}\label{ej:Chap07Ejercicio26}
Dentro de un cilindro adiabático cerrado por un pistón adiabático se encuentra una masa de aire a una presión de $\SI{1}{bar(a)}$ y a una temperatura de $\SI{27}{\celsius}$, ocupando un volumen de $\SI{10}{m^3}$. Se abre una válvula y se introduce una masa $m$ de aire a una presión $\SI{4}{bar(a)}$ y una temperatura de $\SI{50}{\celsius}$. Al introducir esta masa se eleva el pistón, y este choca con unos topes, siendo el volumen final $\SI{20}{m^3}$ y la presión final $\SI{3}{bar(a)}$. Luego de que esto sucede, instantáneamente se cierra la válvula.

Se pide determinar
\begin{enumerate}
    \item La temperatura final y la masa que ingresa.
    \item La variación de entropía del universo.
\end{enumerate}