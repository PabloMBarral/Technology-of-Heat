\section{Ejercicio}\label{ej:Chap07Ejercicio03}
Una máquina térmica intercambia calor con tres fuentes. Las cantidades de calor en valores absolutos son $Q_1=\SI{4000}{kJ}$, $Q_2=\SI{1500}{kJ}$ y $Q_3=\SI{900}{kJ}$.

Las temperaturas de las fuentes son $T_1=\SI{1000}{\kelvin}$, $T_2=\SI{500}{\kelvin}$ y $T_3=\SI{300}{\kelvin}$.

Determinar
\begin{enumerate}
    \item El trabajo $W$ obtenido en la máquina.
    \item El rendimiento térmico $\eta_t$.
\end{enumerate}