\section{Ejercicio}\label{ej:Chap03Ejercicio07}

Al cerrar la puerta de un frezzer, se observa que, si se desea abrirla inmediatamente, hay que hacer una fuerza adicional. El aire del medio ambiente a $\SI{25}{\celsius}$ (supuesto seco) entra al frezzer y, al cerrar la puerta, se enfría hasta $\SI{-22}{\celsius}$. 

Si la puerta tuviera un sellado perfecto, el aire evolucionaría a volumen constante. Calcular la fuerza adicional producida por el
cambio de presión, siendo el área de la puerta de $\SI{0.25}{m^2}$ y el volumen del freezer de $\SI{0.016}{m^3}$. Presión atmosférica $p_0 = \SI{1.01}{bar(a)}$.