\documentclass[10pt,a4paper]{article}

\usepackage[utf8]{inputenc}
\usepackage[spanish]{babel}
\usepackage[top=15mm, bottom=15mm, left=15mm, right=15mm]{geometry}

\usepackage{array}
\usepackage{booktabs}
\usepackage{multirow}

\begin{document}

\begin{center}
     {\huge \textbf{UNIVERSIDAD DE BUENOS AIRES}\\}
          \smallskip
     {\huge \textbf{Facultad de Ingeniería}\\}
          \smallskip
     {\Large \textbf{67.33 Tecnología del Calor}\\}
          \smallskip
     {\Large \textbf{1° cuatrimestre de 2024 - Integradora \#5}}\\
          \bigskip
     {\large Docente: Pablo Barral\footnote{Consultas: \texttt{pbarral@fi.uba.ar}.}}\\
          \bigskip
     {\large 16 de agosto de 2024}\\
\end{center}

\begin{flushleft}
\bigskip
\bigskip
\makebox[12cm]{\textbf{Nombre}:\ \hrulefill}
\bigskip

\makebox[12cm]{\textbf{Padrón}:\ \hrulefill}
\bigskip

\makebox[12cm]{\textbf{Correo electrónico}:\ \hrulefill}
\end{flushleft}
\bigskip
\noindent \rule{\textwidth}{1pt}

\bigskip
\begin{table}[ht!]\label{tab:Puntaje}
\centering
\begin{tabular}[t]{cccccc}
\toprule
\textbf{Ítem}&\textbf{Puntaje}&\textbf{Calificación}&\textbf{Ítem}&\textbf{Puntaje}&\textbf{Calificación}\\
\midrule
1&1,0 pto.&&5.a&1,0 pto.&\\
\midrule
2&1,0 pto.&&5.b&1,0 pto.&\\
\midrule
3&1,0 pto.&&6.a&1,0 pto.&\\
\midrule
4.a&1,0 pto.&&6.b&1,0 pto.&\\
\midrule
4.b&1,0 pto.&&6.c&1,0 pto.&\\
\midrule
&&&\textbf{TOTAL}&\textbf{10,0 ptos.}&\\
\bottomrule
\end{tabular}
\caption{\textit{Distribución del puntaje.}}
\end{table}

\begin{enumerate}

        \item (1 pto.) Esquematizar el corte de una caldera humotubular de dos pasos y fondo húmedo con economizador. Indicar el rango de presión y caudal para este tipo de calderas ¿Por qué el hogar es corrugado?

        \item (1 pto.) ¿Cuáles son las funciones del domo superior en una caldera acuotubular? 
        
        \item (1 pto.) Esquematizar el diagrama $T-Q$ de una caldera paquete tipo $D$ que cuenta con tubos pantalla, sobrecalentador, haz convectivo y economizador. Indicar las temperaturas de los gases y el agua-vapor así como los calores transferidos en cada zona. 

        \item \begin{enumerate}

             \item (1 pto.) Escribir el balance térmico en el hogar, junto con las ecuaciones de transferencia de calor. ¿Qué sucede con la temperatura de llama si la emisividad de esta aumenta? ¿A qué temperatura está la piel de los tubos membrana, aproximadamente?

             \item (1 pto.) ¿Cómo se determinaría el rendimiento de la caldera si supusiéramos que la única pérdida fuera el calor que se llevasen los gases por la chimenea? Escribir qué datos deben medirse y qué ecuación debe aplicarse para determinarlo.
        
        \end{enumerate}
   
         \item \begin{enumerate}
              \item (1 pto.) Esquematizar el diagrama de flujo de una instalación de cogeneración con caldera de recuperación y turbina de vapor a extracción y condensación.

               \item (1 pto.) Para esta instalación, escribir la expresión del rendimiento exergético. ¿Es conveniente realizar una postcombustión en una instalación de este tipo? ¿En qué condiciones sí y en cuáles no, y por qué?
         \end{enumerate}
      
        \item
        \begin{enumerate}
            \item (1 pto.) Esquematizar el diagrama T-Q de una caldera de recuperación en dos presiones, indicando el ó los puntos de pinch y el ó los puntos de approach. Esquematizar el PFD de esa caldera.

            \item (1 pto.) ¿Qué sucede con la producción de vapor de alta presión si la aumento (a la presión)? Indicar esto gráficamente.

            \item (1 pto.) ¿Qué sucede con la producción de vapor de baja presión si introduzco y uso un quemador de conducto en la entrada de la caldera? Indicar esto gráficamente.
        \end{enumerate}  
\end{enumerate}

\end{document}