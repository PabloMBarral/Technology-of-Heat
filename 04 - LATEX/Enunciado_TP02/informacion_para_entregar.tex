\section{Información para entregar}\label{sec:Info}

Una vez realizado el cálculo y consolidados los resultados, se debe presentar para su aprobación un informe completo que recoja el detalle de los pasos realizados, de la información adoptada de los fabricantes de equipos y de las suposiciones asumidas. Esta informe debe permitir al lector reconstruir por su propia cuenta el procedimiento llevado a cabo.

Este informe debe contener, para la situación de referencia, como mínimo la siguiente información:
\begin{enumerate}
    \item PFD\footnote{Diagrama de flujo. Del infglés: Process Flow Diagram.} representando al balance de masa y energía. Marcar en él con claridad el límite de batería de la usina y los flujos de ingreso y de egreso, tanto de materia como de energía. Para cada corriente, debe incluir los parámetros principales: presión, temperatura, entalpía y caudal másico.
    \item Rendimiento térmico y rendimiento exergético de la usina, utilizando los límites de batería indicados en el PFD. Considerar, para este caso, también al consumo del ciclo combinado que abastece a la planta de energía eléctrica.
\end{enumerate}

Y, para cada una de las alternativas, entregar como mínimo:
\begin{enumerate}
    \item PFD incluyendo lo mismo que para la situación de referencia.
    \item En caso de que se utilice una caldera de recuperación, incluir tanto la curva $T-Q$ como los valores de ``pinch point'' y ``approach'' elegidos.
    \item Rendimientos térmicos y exergéticos, de manera similar a la situación de referencia.\footnote{Se sugiere discutir brevemente en clase la expresión que se debe utilizar. De preferencia, usar el concepto ``deltas sobre consumos/deltas''.}
    \item Rendimiento marginal respecto de la situación de referencia.
    \item Ahorro para la industria que instala la cogeneración.
    \item Ahorro para el país por la instalación de la unidad de cogeneración.
    \item Describir y justificar \underline{muy} brevemente qué cambiaría si se permitiese la exportación de energía eléctrica. ¿Qué parámetros deberían ajustarse en la instalación? ¿Habría que cambiar equipos o rediseñar el ciclo? ¿La instalación sería más o menos eficiente?
    \item Describir y justificar \underline{muy} brevemente qué pasaría si se aumentase un $20\%$ la demanda térmica manteniendo la eléctrica constante. ¿Y si sucediese lo mismo con la demanda de energía eléctrica, manteniendo la térmica constante? ¿Y si subiese la demanda térmica, pero bajase la eléctrica? ¿Y si fuera al revés?
\end{enumerate}

Extraer conclusiones. Elegir una de las dos alternativas y justificar en qué se basa esa elección.

\paragraph{Archivos de cálculo:}

En caso de que hayan sido usados scripts o programas, deben incluirse tanto los códigos como los archivos de cálculo. En caso de que sean archivos de \textit{EES}, estos deben entregarse convergiendo.

\paragraph{Unidades:}

En cuanto a unidades, se solicita expresar los caudales molares o volumétricos en $\SI{}{\frac{m^3N}{h}}$ o $\SI{}{\frac{m^3S}{h}}$ según el caso; las presiones en $\SI{}{bar(g)}$; las temperaturas en $\SI{}{\celsius}$; y los caudales másicos en $\SI{}{\frac{kg}{h}}$. Estas son las unidades habituales en la que estos datos son expresados en la práctica.

\paragraph{Presentación de la información:} Se recomienda utilizar cuadros para presentar la información, ya que son muy cómodos para la lectura. Se desaconseja escribir los reemplazos en las fórmulas. Es suficiente incluir las expresiones, los valores asociados y los resultados obtenidos.