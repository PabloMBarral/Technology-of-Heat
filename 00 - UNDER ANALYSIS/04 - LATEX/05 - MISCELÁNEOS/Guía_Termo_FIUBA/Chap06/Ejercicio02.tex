\section{Ejercicio}\label{ej:Chap06Ejercicio02}

Mediante un compresor se lleva una masa de aire de $\SI{100}{\frac{\kg}{h}}$  y una presión de $\SI{100}{kPa(a)}$ y $\SI{20}{\celsius}$ a un estado final con una presión de $\SI{300}{kPa(a)}$. El cigüeñal gira a $\SI{400}{rpm}$. El exponente politrópico de la compresión es $m=1.28$.

Determinar
\begin{enumerate}
    \item El volumen barrido necesario si la relación de espacio nocivo es del $5\%$.
    \item El volumen barrido necesario que debe tener un segundo compresor acoplado al mismo cigüeñal que permite comprimir el aire desde $\SI{300}{kPa(a)}$ hasta $\SI{900}{kPa(a)}$ ingresando el aire a este segundo compresor a $\SI{20}{\celsius}$ produciéndose en el mismo una poltrópica de exponente $m$ igual al del primer compresor, y con la misma relación de espacio nocivo.
    \item La presión intermedia y la masa de aire comprimido en $\SI{}{kg/h}$ si el enfriamiento entre etapas llega hasta $\SI{40}{\celsius}$ en lugar de a $\SI{20}{\celsius}$.
\end{enumerate}