\section{Ejercicio}\label{ej:Chap12Ejercicio10}
Se desea inyectar a una sala, aire a $t_{2(bs)}=\SI{22}{\celsius}$ y $t_{2(bh)}=\SI{17}{\celsius}$ y presión total $p_{2(tot)=\SI{101.325}{kPa(a)}}$ partiendo de aire exterior que está $t_{1(bs)}=\SI{0}{\celsius}$, humedad relativa $\varphi_1=0.8$ ($80\%$) y presión total $p_{1(tot)=\SI{101.325}{kPa(a)}}$. Cómo evolucionaría el aire exterior si disponemos para lograr lo deseado:
\begin{enumerate}[a)]
    \item Un serpentín calentador con el cual podemos calentar el aire exterior.
    \item Un pulverizador para humidificar el aire.
    \item Vapor húmedo a una presión de $\SI{101.325}{kPa(a)}$ y título $x=9.281$.
\end{enumerate}

Se pide calcular en forma analítica (sin datos diagrama):
\begin{enumerate}
    \item La cantidad de calor $Q$ que se entrega en el serpentín calentador.
    \item La temperatura máxima que se alcanza en la etapa de calentamiento ($t_B$).
    \item La cantidad necesaria de vapor $\omega$ para una masa de $\SI{500}{\frac{kg}{h}}$, de aire seco.
    \item Las presiones parciales finales de aire seco y de vapor.
\end{enumerate}