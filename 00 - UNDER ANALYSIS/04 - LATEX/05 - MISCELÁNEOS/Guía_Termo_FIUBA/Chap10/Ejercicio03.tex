\section{Ejercicio}\label{ej:Chap10Ejercicio03}
\lipsum[1]


%Con los mismos datos del ejercicio anterior, pero considerando un rendimiento isoentrópico de 90% en la turbina y en el compresor,
%Determinar
%a) Temperatura a la salida del compresor y de la turbina.
%b) Calor intercambiado en kJ/kg
%c) Trabajo producido por la turbina en kJ/kg.
%d) Trabajo del compresor en kJ/kg
%e) Trabajo neto en kJ/kg
%f) Eficiencia térmica del ciclo.
%g) Calcule la eficiencia térmica del ciclo aumentando la temperatura a la entrada de la
%turbina a 1000K, dejando constante la presión de 500.
%h) Calcule la eficiencia térmica del ciclo aumentando la relación de presión a 6 y dejando constante la temperatura a la entrada a la turbina en 900K