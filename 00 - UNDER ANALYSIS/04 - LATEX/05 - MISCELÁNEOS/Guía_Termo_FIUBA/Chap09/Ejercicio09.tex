\section{Ejercicio}\label{ej:Chap09Ejercicio09}

Una masa de agua se encuentra dentro de un cilindro cerrado por un pistón que puede deslizarse sin rozamiento y que tiene dos topes: uno de máximo desplazamiento y otro de mínimo. El volumen máximo es de $V_{\text{MÁX}}=\SI{20}{m^3}$ y el mínimo es de $V_{\text{MÍN}}=\SI{1}{m^3}$.

El pistón tiene un peso de $G=\SI{5000}{kg_f}$ y su área es $A=\SI{0.5}{m^2}$. La presión del medio es $p_0=\SI{101.3}{kPa(a)}$ y su tempeeratura es de $t_0=\SI{25}{\celsius}$.

Inicialmente el agua está a $t_1=\SI{20}{\celsius}$, el líquido ocupa un volumen $V_L=\SI{0.01}{m^3}$ y el del vapor $V_V=\SI{0.99}{m^3}$.

Desde una fuente externa se entrega calor al agua hasta que esta alcanza $\SI{473}{\kelvin}$. El único intercambio de calor es el de la fuente con el agua.

Se pide
\begin{enumerate}
    \item Hacer un esquema del dispositivo.
    \item Determinar el estado inicial del agua: masa, título y presión.
    \item Determinar el estado final del agua: volumen, presión, temperatura, coeficiente de compresibilidad y exergía.
    \item Dibujar un diagrama $p-v$, indicando los estados del agua y su evolución. Repetir para un diagrama $T-s$.
    \item El calor y el trabajo transferidos por el agua.
\end{enumerate}