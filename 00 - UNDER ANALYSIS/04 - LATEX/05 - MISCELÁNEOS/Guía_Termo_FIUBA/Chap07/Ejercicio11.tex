\section{Ejercicio}\label{ej:Chap07Ejercicio11}
Determinar los cambios de entropía sufridos por los siguientes sistemas termodinámicos
\begin{enumerate}
    \item Una fuente térmica con una temperatura $T_F=\SI{600}{\celsius}$ que recibe $\SI{500}{MJ}$ de calor.
    \item Una fuente térmica con una temperatura $T_F=\SI{600}{\celsius}$ que cede $\SI{500}{MJ}$ de calor.
    \item Un cuerpo de $\SI{6}{kg}$ de calor específico constante $c=\SI{2.5}{\frac{kJ}{kg\,\kelvin}}$ que pierde $\SI{500}{MJ}$ e inicialmente tiene una temperatura de $\SI{150}{\celsius}$. Trazar a mano alzada la curva del enfriamiento del cuerpo en el diagrama $T-s$.
    \item Agua líquida que describe un ciclo termodinámico, pasando por estados de alta presión y temperatura, estados de vapor sobrecalentado y de hielo (sólido), partiendo y llegando a la misma presión y temperatura que la inicial.
\end{enumerate}