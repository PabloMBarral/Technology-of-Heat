\section{Ejercicio}\label{ej:Chap07Ejercicio23}
Una masa de $\SI{10}{kg}$ de aire contenida en un recinto rígido y adiabático, inicialmente a una temperatura de $\SI{300}{\kelvin}$ y a una presión de $\SI{1}{bar(a)}$, se lleva a una temperatura de $\SI{500}{\kelvin}$ de diferentes modos.

Determinar, para cada uno
\begin{enumerate}
    \item El calor aportado.
    \item La variación de energía interna.
    \item El trabajo suministrado.
    \item Las variaciones de entropía de medio, sistema y universo.
\end{enumerate}

Los casos son
\begin{enumerate}[A.]
    \item Mediante trabajo aportado por un agitador.
    \item Mediante una resistencia eléctrica.
    \item $50\%$ de la energía mediante un agitador y el otro $50\%$ con una fuente a $\SI{500}{\kelvin}$.
    \item Con una fuente a $\SI{500}{\kelvin}$.
    \item Con una fuente a $\SI{600}{\kelvin}$.
    \item $50\%$ de la energía con una fuente a $\SI{400}{\kelvin}$ y el otro $50\%$ con una fuente a $\SI{500}{\kelvin}$.
    \item Con tres fuentes, a $\SI{380}{\kelvin}$, $\SI{450}{\kelvin}$ y $\SI{500}{\kelvin}$.
    \item Con infinitas fuentes a temperaturas crecientes.
\end{enumerate}