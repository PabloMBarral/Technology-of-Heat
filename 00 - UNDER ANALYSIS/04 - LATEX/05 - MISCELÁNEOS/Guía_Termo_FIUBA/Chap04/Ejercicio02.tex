\section{Ejercicio}\label{ej:Chap04Ejercicio02}
Escribir la expresión del trabajo realizado por
\begin{enumerate}
    \item Un sistema cerrado de masa $m$ que evoluciona cuasiestáticamente al aumentar su volumen de $V_1$ hasta $V_2$.
    \item Un sistema cerrado de masa $m$ sometido a una presión constante $p_1$ al aumentar su volumen de $V_1$ hasta $V_2$.
    \item Un sistema abierto en régimen estacionario con masa $\dot{m}$, de entrada $e$ y salida $s$, que evoluciona cuasiestáticamente al aumentar su volumen específico de $v_e$ hasta $v_s$, su presión de $p_e$ a $p_s$, su velocidad de $\omega_e$ a $\omega_s$ y su altura de $z_e$ a $z_s$.
    \item Un sistema abierto en régimen estacionario con masa $\dot{m}$, que evoluciona cuasiestáticamente al aumentar su volumen específico de $v_1$ hasta $v_2$ a presión constante, sin cambios de energía cinética ni potencial gravitatoria.
    \item Un sistema abierto en régimen estacionario, con masa $\dot{m}$ que evoluciona cuasiestáticamente al aumentar su presión de $p_1$ hasta $p_2$, a volumen específico constante, sin cambios de energía cinética ni potencial gravitatoria.
    \item Un sistema abierto en régimen estacionario, con masa $\dot{m}$, no cuasiestático al aumentar su volumen específico de $v_1$ hasta $v_2$ a presión constante.
    \item Un sistema en régimen transitorio que evoluciona cuasiestáticamente sin cambios de energía cinética ni potencial gravitatoria.
\end{enumerate}