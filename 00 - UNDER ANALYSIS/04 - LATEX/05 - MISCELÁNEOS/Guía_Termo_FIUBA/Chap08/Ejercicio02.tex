\section{Ejercicio}\label{ej:Chap08Ejercicio02}
Para un estado de referencia $T_0=\SI{300}{\kelvin}$ y $p_0=\SI{1}{bar(a)}$, determinar la exergía de los siguientes casos
\begin{enumerate}
    \item $\SI{100}{kg}$ de aire (gas ideal), a $t_1=\SI{650}{\celsius}$ y $p_1=\SI{12}{bar(a)}$.
    \item $\SI{100}{kg}$ de aire (gas ideal), a $t_1=\SI{27}{\celsius}$ y $p_1=\SI{0.1}{bar(a)}$.
    \item $\SI{7d6}{m^3}$ de agua de un lago a $\SI{70}{m}$ de altura sobre el nivel de referencia, a una presión $p_0$ y temperatura $T_0$.
    \item Una bala de cañón de $\SI{10}{kg}$, que se desplaza horizontalmente a $\SI{250}{\frac{m}{s}}$, con rozamiento y sin rozamiento.
    \item Una corriente de $\SI{60}{\frac{kg}{h}}$ de oxígeno en régimen permanente, a $p_1=\SI{12}{bar(a)}$ y $t_1=\SI{-20}{\celsius}$.
    \item Una corriente de $\SI{30}{\frac{kg}{h}}$ de hidrógeno a $p_1=\SI{0.01}{bar(a)}$ y $T_1=\SI{300}{\kelvin}$.
    \item La exergía de $\SI{10000}{kcal}$ de energía térmica que salen de una estufa a $\SI{600}{\celsius}$.
    \item La exergía de $\SI{10000}{kcal}$ de energía térmica que ingresan a un salón a $\SI{40}{\celsius}$.
    \item La exergía de $\SI{1}{kWh}$ de trabajo eléctrico entregado a una resistencia eléctrica.
    \item La exergía de $\SI{1}{m^3}$ de vacío.
\end{enumerate}