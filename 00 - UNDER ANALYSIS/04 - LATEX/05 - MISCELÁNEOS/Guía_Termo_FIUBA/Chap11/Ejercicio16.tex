\section{Ejercicio}\label{ej:Chap11Ejercicio16}
Para refrigerar un ambiente se elige un ciclo de compresión con las siguientes características
\begin{itemize}
    \item Calor por retirar del ambiente: $\dot{Q}_2=\SI{4500}{kcal}$.
    \item Temperatura de evaporación: $t_2=\SI{7}{\celsius}$.
    \item Temperatura de condensación: $t_1=\SI{45}{\celsius}$.
    \item Aspiración del compresor: vapor sobrecalentado a $t_3=\SI{9}{\celsius}$.
    \item Rendimiento isoentrópico del compresor: $\eta_{iso}=0.85$.
    \item Ingreso a la válvula de expansión: líquido saturado proveniente del condensador.
\end{itemize}

Se pide, usando como fluido intermediario agua, en un caso, y refrigerante $R12$ en el otro,
\begin{enumerate}
    \item Dibujar un esquema de la instalación, numerarlo de acuerdo a lo expuesto y dibujar el diagrama $T-s$ correspondiente.
    \item Generar una tabla con los valores de $p$, $T$, $h$, $s$, $x$ y $v$ de los diferentes estados del ciclo.
    \item Temperatura máxima del ciclo.
    \item Presiones máxima y mínima del ciclo.
    \item Gasto másico $\dot{m}$.
    \item Volumen específico máximo del ciclo $v$.
    \item Potencia necesaria para accionar el compresor $\dot{W}$.
    \item Coeficiente de efecto frigorífico.
    \item Si el compresor fuera del tipo alternativo y su eje girara a $\SI{1500}{rpm}$ (revoluciones por minuto), ¿cuál sería el volumen barrido por el pistón en la etapa de admisión, y cuál sería su diámetro? Considerar que la carrera es igual al diámetro del pistón.
    \item Comparar en una tabla las respuestas a los puntos $3$ a $9$ para ambos fluidos circulantes y extraer conclusiones.
\end{enumerate}