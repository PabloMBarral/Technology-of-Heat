\section{Ejercicio}\label{ej:Chap11Ejercicio17}
Un sistema de refrigeración opera con aire como sustancia de trabajo. Aspira aire exterior a $t_1 = \SI{35}{\celsius}$ y $\SI{101.3}{kPa(a)}$ y lo comprime isoentrópicamente hasta $\SI{150}{kPa(a)}$. El aire comprimido se enfría hasta $t_3 = \SI{55}{\celsius}$ en un intercambiador de superficie y, finalmente, se lo expande isoentrópicamente hasta $\SI{101.3}{kPa(a)}$, inyectando este aire al local a refrigerar.

Determinar:
\begin{enumerate}
    \item Esquema de la instalación.
    \item Diagrama $T-s$ correspondiente.
    \item Temperatura de inyección al local $t_4$.
    \item Coeficiente de efecto frigorífico $COP$.
    \item Para un efecto frigorífico de $\dot{Q}_F=\SI{12}{kW}$, calcular la masa de aire horaria, la potencia de accionamiento y el volumen de aire de aspiración en $\SI{}{\frac{m^3}{h}}$. Expresar el efecto frigorífico en toneladas de refrigeración.
    \item Trazar el diagrama $T-s$ correspondiente al mismo ciclo pero con irreversibilidades en la compresión y en la expansión, y explicar cómo se modifica la temperatura de inyección al local, la potencia de accionamiento y el COP, justificándolo.
\end{enumerate}