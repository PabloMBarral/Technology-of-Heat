\section{Ejercicio}\label{ej:Chap03Ejercicio01}

Usando la ecuación de estado de equilibrio de los gases ideales en sus dos formas
\begin{gather}
    p\,V = m_{O_2}\,R_{p_{O_2}}\,T\\
    p\,V = n_{O_2}\,R_u\,T
\end{gather}
determinar el volumen $V\,[\SI{}{m^3}]$ y el número de moles $n_{O_2}$ de una masa de oxígeno $m_{O_2}=\SI{20}{\kg}$ que se encuentra a una temperatura de $\SI{-25}{\celsius}$ y a una presión de $\SI{5}{bar(a)}$.

Utilizar la constante universal de los gases ideales y la constante particular del oxígeno
\begin{gather}
    R_u=\SI{8.314}{\frac{kJ}{\kg\,\K}}\\
    R_{p_{O_2}} = \frac{R_u}{MW_{O_2}}
\end{gather}
siendo 
\begin{equation}
    MW_{O_2}=\frac{m_{O_2}}{n_{O_2}}=\SI{32}{\frac{\kg}{\kmol}}
\end{equation}
la masa de un $\SI{}{\kmol}$ de moléculas de oxígeno.