\section{Ejercicio}\label{ej:Chap03Ejercicio06}
El calor específico a presión constante de un gas es considerado, en el modelo de gas ideal, como constante ante cambios de temperatura.\footnote{A veces se suele denominar a esta combinación como gas perfecto.} En general, este calor específico es una función de la temperatura.

Para el aire, el calor específico a presión constante para una presión de $\SI{101}{kPa(a)}$ se puede calcular como
\begin{multline}
    c_{p_a}\left(T\left[\SI{}{\kelvin}\right]\right)\,\left[\SI{}{\frac{kJ}{kmol\,\kelvin}}\right]=\num{-2.122d-10}\,T^3 \\-\num{1.035d-6}\,T^2
    +\num{6.906d-3}\,T + 28.54
\end{multline}

Calcular el valor medio de $c_{p_a}$ entre $\SI{100}{\celsius}$ y $\SI{800}{\celsius}$, y expresarlo en $\SI{}{\frac{kJ}{kg\,\kelvin}}$. ¿Cuál es el error relativo y el error absoluto que se comete al aproximar el calor específico a $\SI{1800}{\kelvin}$ con el valor a $\SI{273}{\kelvin}$?