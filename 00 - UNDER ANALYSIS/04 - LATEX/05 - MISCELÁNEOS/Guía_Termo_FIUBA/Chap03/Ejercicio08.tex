\section{Ejercicio}\label{ej:Chap03Ejercicio08}
Un recipiente cerrado de $\SI{2}{\meter^3}$ de capacidad está dividido en dos partes iguales por una membrana semipermeable que permite el paso del gas $A$ e impide el del gas $B$.

Inicialmente en el recipiente se ha hecho vacío. Del lado $1$ se inyecta $\SI{1}{\kg}$ de gas $A$ y del $2$ $\SI{1}{\kg}$ de gas $B$. 

Luego de un tiempo, se establece el equilibrio, y la temperatura del conjunto resulta $\SI{30}{\celsius}$.

Calcular la presión total a ambos lados del recinto.

Constantes de los gases (ideales) 
\begin{align*}
    R_A&=\SI{0.300}{\frac{kJ}{kg\,K}}\\
    R_B&=\SI{0.400}{\frac{kJ}{kg\,K}}
\end{align*}